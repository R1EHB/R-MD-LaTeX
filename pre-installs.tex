\documentclass{article}
\usepackage[utf8]{inputenc}
\usepackage{hyperref}

\title{R/LaTeX/Markdown workshop pre-requisites and EPA installation instructions}
\author{Emily Li and Erik Beck}
\date{2018}

\begin{document}
\maketitle
%\tableofcontents

\section{Basic info about the workshop}
\begin{center}
    \begin{tabular}{c c c}
    Day & Time & Room \\
    \textbf{Tues, Sept 11} & \textbf{8:30-11:30am} & \textbf{C111C}
\end{tabular}
\end{center}

This 1/2-day workshop will provide attendees with hands-on experience
using the basics of LaTeX, Markdown, and the R package knitr. After
attending this workshop, you will be able to use these tools to
facilitate reproducible reports and research with R.

\section{You will need}
It is expected that you have some familiarity with R and RStudio, can
install packages from CRAN, and understand basic R syntax.

We will be using R, RStudio, and MiKTeX during this workshop. These must be installed prior to September 11.
\begin{itemize}
    \item R, RStudio
    \item MiKTeX
    %\item Your preferred TeX editor (TeXStudio, TeXWorks, ShareLaTeX, emacs, notepad, notepad++, etc.)
    \item R packages:
    \begin{itemize}
        \item knitr
        \item ggplot2 (or all of tidyverse)
        \item pandoc
    \end{itemize}
\end{itemize}

\subsection{R and RStudio}
Simply email EZTech, now ``EISD", at
\href{mailto:EISD@epa.gov}{EISD@epa.gov}) and request that R and
RStudio be installed on your machine. They will usually push that out
within a day or so.

\subsection{MiKTeX}
\begin{enumerate}
    \item Go to \url{https://usepa.sharepoint.com/sites/EZForms/SitePages/Home.aspx}
    \item Under Create a Request, click on ''Freeware/Shareware''
    \item Fill out your employee info
    \item For ``Justification'', say: Needed in conjunction with R and
      associated packages to create self-contained documentation and
      reports using R. This makes the paradigm of reproducible
      research feasible with data analysis. MiKTeX is a bundling of
      the TeX and LaTeX typesetting tools for Windows OS. The website
      for MiKTeX base download and installation is:
      https://miktex.org/download
    \item For ``Software Title'', say: MiKTeX
    \item Enter supervisor name
\end{enumerate}

%\subection{Any TeX editor or compiler}
%There are \href{https://www.sharelatex.com/learn/Choosing_a_LaTeX_Compiler}{dozens of online and offline TeX editors} that also compile documents from LaTeX code. Popular online options are \href{https://www.sharelatex.com/}{ShareLaTeX} and \href{https://www.overleaf.com/}{Overleaf}. We should have internet connection during the workshop, but you may want to be familiar with offline options you can have installed to your machine.
%
%A popular offline option is TeXStudio.
%
%Worst case scenario, if you are unable to have a compiler installed, Notepad comes standard with Windows OS and can edit source code.

% Any good text editor will do.
% Word is not a great text editor.
% Notepad is workable.
% Good online article about how text editor and 
% wp are different: 
% https://www.howtogeek.com/299490/whats-the-difference-between-notepad-and-wordpad-in-windows/

\subsection{R packages}
We will be using particular packages/libraries during the workshop that you will need to install in RStudio. You can follow the instructions below to do so:

\begin{enumerate}
    \item Run R studio
    \item Click on the Packages tab in the bottom-right section and then click on install. A dialog box will appear.
    \item In the Install Packages dialog, write the package name you want to install under the Packages field and then click install. This will install the package you searched for or give you a list of matching package based on your package text.
\end{enumerate}

You will need to install these packages, if they are not already installed:
\begin{itemize}
    \item knitr
    \item ggplot2
    \item pandoc
\end{itemize}

\end{document}
